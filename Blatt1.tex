\documentclass{article}
\usepackage{myStyle}
\usepackage{commands}

\begin{document}

\section{Operatornorm}
Für einen Operator $A \colon V \to W$ für zwei normierte Räume $V$ und $W$ ist die Operatornorm definiert durch
$$\| A \|:= \sup_{\| u \|_V = 1} \| A u \|_W.$$
Wir betrachten $V = W = C([0,1])$ mit
$$\|u \|_{C([0,1])} = \| u \|_\infty = \sup_{t\in [0,1]} |u(t)|.$$
Sei $u \in C([0,1])$ mit $\| u \|_\infty = 1$. Es gilt
\begin{align*}
    |\left(Au\right)(t)| & \le \int_0^1 |e^{t-s} u(s) |~ds\\
    &= \int_0^1 e^{t-s} |u(s)| ~ds \\
    &\stackrel{t\in [0,1]}\le \underbrace{\|u\|_\infty}_{=1} \int_0^1 e^{1-s}~ds \\
    &= e\int_0^1 e^{-s}~ds = e-1 
\end{align*}
Für die konstante Funktion $u \equiv 1$ gilt Gleichheit, denn dann ist
$$\| Au \|_\infty = \sup_{t\in [0,1]} \int_0^1 e^{t-s}~ds = e-1$$
und somit folgt auch
$$\| A \| = e-1.$$

\section{Stetiger Operator}
Es gilt
\begin{align*}\| Tx \|_2 &= \left| \sum_{i=1}^\infty \frac{ξ_i}{i} \right| \le \sum_{i=1}^\infty \left| \frac{ξ_i}{i}\right| \\
&\stackrel{\text{Hölder}}\le \left\| \left( \frac{1}{i} \right)_{i\in ℕ} \right\|_2 \cdot \|x \|_2
\end{align*}
Daraus folgt $\| T \|<\infty$.
Einsetzen der gegebenen Folge gilt
$$\|Tx_1 \|_\infty = \sum_{i=1}^n \frac{1}{i}\to \infty\quad \text{für $n\to \infty$}$$
und also in dieser Norm $\| T \|_{infty} = \infty$.

\setcounter{section}{2}
\section{schwach konvergent impliziert stark konvergent}
Sei $H$ ein Hilbertraum mit Skalarprodukt $\langle •,•\rangle$. Sei $u_n$ eine stark konvergente Folge, also $\|u_n - u\|_H \to 0$ für $n\to \infty$. Für $φ\in H^*$ gilt dann
$$|φ(u_n) - φ(u) | = |φ(u_n - u)| \le \| φ \|_{H^*} \|u_n - u \|_H \to 0,$$
also 
$$\lim_{n\to \infty} φ(u_n) = φ(u).$$

\section{Konvergenzbegriffe}
Es sei $H$ ein Hilbertraum mit Skalarprodukt $\langle •,•\rangle$. 
Es konvergiere die Folge $u_n$ schwach gegen $u$, also
$$u_n \wto u :⇔ \lim_{n\to \infty} φ(u_n) = φ(u) \quad ∀φ\in H^*$$
Mit dem Rieszschen Darstellungssatz gibt es für jedes $φ\in H^*$ genau ein $w\in H$, so dass 
$$φ(u_n) = \langle u_n, w \rangle\qquad ∀u_n \in H,$$
also erhalten wir
$$u_n \wto u ⇔ \lim_{n\to \infty} \langle u_n, w \rangle = \langle u, w \rangle \quad ∀w\in H.$$
Dann gilt außerdem
$$\lim \langle u_n -u, w \rangle = 0 \qquad ∀w\in H.$$
Jede schwach konvergente Folge ist beschränkt, es gibt also ein $M\ge 0$, so dass
$$\|u_n\|_H \le M\qquad ∀n\in ℕ.$$

Ist ferner $v_n \to v$, gilt also
$$\lim_{n\to \infty} \|v_n - v \|_H = 0,$$
so folgt
\begin{align*}
    |\langle u_n, v_n \rangle - \langle u, v \rangle | &= | \langle u_n, v_n \rangle - \langle u_n, v \rangle + \langle u_n, v\rangle - \langle u, v\rangle |\\
    &= |\langle u_n , v_n - v \rangle  -\langle u - u_n , v\rangle| \\
    &\le |\langle u_n, v_n - v \rangle | + |\langle u - u_n, v \rangle |\\
    &\stackrel{\text{C.S.}}\le \underbrace{\| u_n \|_H}_{\le M} \underbrace{\|v_n - v\|_H}_{\to 0} + \underbrace{| \langle u - u_n, v \rangle |}_{\to 0} \to 0 
\end{align*}
\end{document}