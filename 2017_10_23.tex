\documentclass[main.tex]{subfiles}
% vl 23.10.18

\begin{document}

\setcounter{chapter}{2}
\setcounter{satz}{5}
Nach Einführung dieser Begriffe, stellt sich die Frage: Welche Funktionen sind \suhs{}?
In der Übung leitet wir desweiteren Rechenregeln für \suhs{}e Funktionen her.
Man beobachtet, dass nicht alle stetigen Funktionen schwach unterhalb stetig sind:

\begin{bsp}\label{2.6}$ $\\[-1em]
\begin{itemize}
\item Im Endlichdimenisonalen, z.B. für $F\colon ℝ\to ℝ$ folgt aus $F$ stetig, dass $F$ auch \suhs{}.
\item Wir untersuchen ein Gegenbeispiel im Unendlichdimensionalen:\\
Sei $X = L^2([0,2π]) = M$ und 
$$f\colon X\to ℝ, f(x) = -\int_0^{2π} \left(x(t)\right)^2~dt = -\| x\|_{L^2}^2.$$
Offensichtlich ist $f$ stetig.
Wir betrachten die Folge
$$x_n(t) := \frac{1}{\sqrt{π}} \sin (nt), \qquad n\in ℕ.$$
Für $g\in L^2([0,2π])$ folgt, dass
$$\int_0^{2π} g(t) \frac{1}{\sqrt{π}}\sin(nt)~dt$$
der $n$-te Fourier-Koeffizient bzgl. des Orthonormalsystems $\left\{ \frac{1}{\sqrt{π}}\sin(nt)\right\}$ ist. Dann folgt aber auch
$$\int_0^{2π} g(t) \frac{1}{\sqrt{π}} \sin (nt)~dt \to 0.$$
Da $g\in L^2([0,2π])$ beliebig war folgt mit Lax-Milgram $x_n \wto 0$.
Aber es gilt $f(x_n) = -1$ und $f(0) = 0$, also
$$\liminf_{n\to ∞} f(x_n) = -1 < 0$$
und $f$ ist nicht \suhs{}.  
\end{itemize}
\end{bsp}

\subsection*{Welche zusätzlichen Eigenschaften sichern schwache Unterhalb-Stetigkeit?}

\begin{mydef}\label{2.7}
Es sei $X$ ein reeller Vektorraum und $M\subseteq X$.
Die Menge $M$ heißt konvex, wenn für $x,y\in M$ und $λ\in [0,1]$ gilt
$$λx + (1-λ) y \in M\qquad ∀λ\in [0,1].$$
% standardskizze: konvexe Menge und nicht-konvexe Menge
Sei $M\ne ∅$ und konvex. Dann heißt eine Funktion $f\colon M\to ℝ$ konvex, wenn
$$f(λx + (1-λ)y)) \le λf(x) + (1-λ) f(y) \qquad ∀ x,y\in M, ∀λ\in [0,1].$$
% standardskizze: konvexe Funktion.
\end{mydef}

\begin{bsp}\label{2.8}$ $\\[-1em]
\begin{itemize}
\item Sei $X = C^1([0,1]), M = C_0^1([0,1])$. Dann ist $M$ eine konvexe Menge, denn für $x,y \in M$ ist auch $λx + (1-λ)y \in C^1([0,1])$ und am Rand gilt
$$\left( λx + (1-λ) y \right)(0) = λx(0) + (1-λ) y(0) = 0$$
(1 analog), also $λx + (1-λ) y \in C_0^1 \left( [0,1] \right).$
\item Sei $X$ ein reeller Vektorraum. Dann ist jede Norm auf $X$ eine konvexe Funktion,
denn 
$$\| λx + (1-λ) y \| \le \| \underbrace{λ}_{\ge 0} x \| + \| \underbrace{(1-λ)}_{\ge 0} y \| = λ\| x\| + (1-λ) \| y\|, \qquad λ\in [0,1].$$ 
\end{itemize}
\end{bsp}

\begin{satz}\label{2.9}
Sei $(X, \|•\|)$ ein normierter Raum, $M\subseteq X, M\ne ∅$ und $M$ konvex. Sei ferner $f\colon M\to ℝ.$ Dann sind äquivalent:
\begin{enumerate}[label=(\roman*)]
\item \label{2.9.1} $f$ ist konvex.
\item \label{2.9.2} $E(f)$ ist konvex.
\end{enumerate}
Außerdem folgt aus einer dieser Aussagen, dass
\begin{enumerate}[label=(\roman*)]
\setcounter{enumi}{2}
\item \label{2.9.3} $M_α = \{ x\in M: f(x) \le α \}$ ist für jedes $α\in ℝ$ konvex. 
\end{enumerate}
\end{satz}

\begin{proof}
Wir zeigen zunächst \ref{2.9.1} $\Rightarrow$ \ref{2.9.2}. Sei dazu $f$ konvex. Dann gilt für $(x,α),(y,β)\in E(f)$ und $λ\in [0,1]$:
\begin{align*}
f(λx+ (1-λ) y) &\stackrel{\text{$f$ konvex}}\le λf(x) + (1-λ) f(y)\\
&\stackrel{(x,α), (y,β)\in E(f)}\le λα + (1-λ) β
\end{align*}
und somit 
$$λ(x, α) + (1-λ) (y,β) \in E(f) \qquad ∀λ\in [0,1].$$
Also ist $E(f)$ konvex.

Jetzt gelte \ref{2.9.2}, es sei also $E(f)$ konvex. Wir zeigen \ref{2.9.3}. 
Sei $α\in ℝ$ so, dass $M_α\ne ∅$. Dann gilt für $x,y \in M_α$ und $λ\in [0,1]$, dass
$$λ(x,α) + (1-λ) (y,α)\in E(f)$$
und damit 
$$f(λx + (1-λ) y ) \le λα + (1-λ) α = α.$$
Daraus folgt
$λx + (1-λ) y \in M_α$ und $M_α$ ist also konvex.

Schließlich folgern wir \ref{2.9.1} aus \ref{2.9.2}. Sei $E(f)$ konvex. Dann gilt für $x,y\in M$ und $λ\in [0,1]$ wegen $λ(x,f(x)) + (1-λ) (y,f(y))\in E(f)$, dass
$$f(λx + (1-λ)y) \le λf(x) + (1-λ) f(y).$$
Also ist $f$ konvex.
\end{proof}

\begin{bem*}
Aus der Konvexität des Levelsets $M_α$ folgt \emph{nicht} die Konvexität von $f$.
\end{bem*}

\begin{bsp}\label{2.10}
Sei $f\colon ℝ\to ℝ$ mit
$$f(x) = \begin{cases}
(x-2)^2 &\text{für $x<2$},\\
(x-2)^2 + 1 &\text{für $x\ge 2$}.\end{cases}$$
% skizze plot 
Offensichtlich ist $M_α = [\underline x_α, \ov x_α]$ für $α\ge 1$ und $M_α = [\underline x_α, \ov x_α )$ für $α<1$. Intervalle in $ℝ$ sind Konvex, aber $f$ ist nicht konvex.

Sei alternativ
$$f\colon ℝ\to ℝ, f(x) = \begin{cases}
(x-2)^2 + 1 &\text{für $x>2$},\\
-(x-2)^2 +1 &\text{für $x\le 2$}.\end{cases}$$
%skizze plot
Dann gilt $M_α = (-∞ ,\ov x_α]$ ist konvex, $f$ stetig, aber nicht konvex.
\end{bsp}

Diese Beispiele motivieren entsprechende Verallgemeinerungen:
\begin{mydef}\label{2.11}[Quasikonvexität]
Sei $(X, \|•\|)$ ein normierter, linearer Raum, $M\subset X, M\ne ∅$ und konvex und $f\colon M\to ℝ$. Ist für jedes $α\in ℝ$ die Menge $M_α := \{ x\in M: f(x) \le α\}$ konvex, so heißt $f$ \bemph{quasikonvex}.
\end{mydef}

\begin{lemma}\label{2.12} Sei $(X, \| •\|)$ ein normierter Raum, $M\subset X, M\ne ∅$ konvex und abgeschlossen. Ist $f\colon M\to ℝ$ stetig und quasikonvex, dann ist $f$ \suhs{}.
\end{lemma}

\begin{proof}[Beweisskizze]
Sei $α \in ℝ$ beliebig gewählt.
Für $M_α := \{ x\in M: f(x) \le α\} \ne ∅$ ist $M_α$ auch abgeschlossen, da $f$ stetig und $M$ abgeschlossen ist.\\
Da $f$ quasikonvex ist, ist $M_α$ konvex. es gilt
\begin{lemma*}Für $X$ wie in Lemma \ref{2.12} gilt:
Jede konvexe und abgeschlossene Menge ist schwach folgenabgeschlossen.\\
(vgl. Übung oder \cite{altLinear}.)
\end{lemma*}
Dann folgt aus \ref{2.5}, dass $f$ \suhs{} ist.
\end{proof}

\begin{mydef}\label{2.13}
Banachraum/Hilbertraum
\end{mydef}
\begin{bsp*}
$\left( C([a,b]), \|	• \|_\infty \right)$ ist ein Banachraum, aber $\left( C([a,b]), \|•\|_{L^2} \right)$ ist \emph{kein} Banachraum.
\end{bsp*}


\begin{mydef}\label{2.14}
Sei $(X, \| 	• \|)$ ein reeller, normierter, linearer Raum. Die Menge aller linearen, stetigen Funktionen $f\colon X\to ℝ$ heißt der zu $X$ duale Raum. Er wird mit $\mathcal L(X, ℝ)$ oder $X^*$ bezeichnet. Die zugehörige Norm ist
$$\| f\|_{X^*} = \sup_{\|x\|_X = 1} | f(x)|.$$
\end{mydef}
\begin{bem*}
Wegen der Vollständigkeit der reellen Zahlen ist $X^*$ stets ein Banachraum.
\end{bem*}

\begin{bsp}\label{2.15}
Sei $(X, \|•\|) = \left( C([0,1]), \|•\|_∞ \right)$. Dann gilt für $f\in X^*$ mit $f(x) = x \left( \frac{1}{2} \right)$, dass
\begin{itemize}
\item $f\colon X \to ℝ$ linear,
\item $$|f(x)| = \left| x\left( \frac{1}{2} \right)\right|\le \max_{t\in [0,1]} |x(t)| = 1\cdot \| x \|_∞.$$
Also $\| f\|_{X^*} \le 1.$
Für $x(t) \equiv 1$ gilt $|f(x)| = \left| x\left( \frac{1}{2}\right) \right| = 1 = \| x\|_∞$, also $\|f\|_{X^*} \ge 1$, also $\| f\|_{X^*} = 1$ und $f$ ist stetig und $f\in X^*$. 
\end{itemize}
\end{bsp}

Sei $X$ ein reeller Banachraum mit Dualraum $X^*$. Ist $x\in X$ ein fest gewähltes Element und wird $f\in X^*$ als variabel betrachtet, so ist die durch 
$$F_x\colon X^* \to ℝ, \; F_x(f) = f(x)\in ℝ$$
definierte Abbildung linear und stetig, wegen
$$|F_x(f)| = |f(x)|\le \| f\|_{X^*} \| x\|_X,$$
und daher $F_x\in \left( X^* \right)^* =: X^{**}.$
Der Raum $X^{**}$ heißt Bidualraum zu $X$. Mit der Zuordnung $x\to F_x$ als kanonische Einbettung gilt immer $X\subset X^{**}.$

\begin{mydef}
Sei $(X, \| • \|)$ ein reeller, normierter, linearer Raum und $X^*$ der zugehörige Dualraum und $X^*$ der zugehörige Dualraum. Ist die kanonische Einbettung $X \to X^{**}, x\mapsto F_x$ surjektiv, d.h. $X=X^{**}$, so heißt $X$ reflexiv.
\end{mydef}
D.h. bei einem reflexiven Banachraum erhält man bei zweimaliger Dualisierung wieder den ursprünglichen Raum.

\begin{mydef}\label{2.17}
Sei $Ω\subset ℝ^n, n\in ℕ$ eine beschränkte, Lebesgue-messbare Menge und $1\le p < ∞$. Dann definiert man
$$L^p(Ω) = \left\{ y:Ω\to ℝ: \int_Ω |y(t)|^p~dt < ∞ \right\}$$
mit 
$$\| y\|_{L^p} = \left( \int_Ω |y(t)|^p ~dt \right)^{\frac{1}{p}}.$$
$L^p(Ω)$ ist ein reeller, normierter, linearer Raum.
\end{mydef}

\begin{mydef}\label{2.18}
Sei $Ω \subset ℝ^n$ wie in Definition \ref{2.17}. Dann definiert man
$$L^∞(Ω) := \left\{ y\colon Ω\to ℝ :\text{ $y$ ist fast überall gleichmäßig beschränkt und messbar}\right\}.$$
Mit
$$\| y\|_{L^∞} = \esssup_{x\in Ω} |y(x)| := \inf_{|E|=0} \left( \sup_{x\in Ω\setminus E} |y(x)| \right)$$
wird $L^∞(Ω)$ zu einem reellen, normierten, linearen Raum.
\end{mydef}
\end{document}