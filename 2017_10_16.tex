\documentclass[main.tex]{subfiles}
\begin{document}
\setcounter{chapter}{1}
\chapter{Existenz von Lösungen}
Für ein gegebenes Optimierungsproblem ist im Allgemeinen nicht klar, ob überhaupt Lösungen existieren. Zum Beispiel ist
\begin{align*}
    &\min_{x\in X} f(x) = \min_{x\in X} \exp(-2x^3),\\
    &X = M = ℝ
\end{align*}
nicht lösbar. Daher benötigt man Kriterien für $f$ und $M$, welche die EXistenz von Lösungen sichern.

\section{Existenzaussagen}
\begin{satz*}[Satz von Weierstraß]
Jede stetige Funktion $f:ℝ^n \to ℝ$ nimmt auf einer kompakten Teilmenge ein Minimum an.
\end{satz*}
Dieser Satz gilt \bemph{nicht} im Unendlichdimensionalen. In diesem Fall nehmen wir zunächst an, $f$ ist stetig und $M$ kompakt. Ferner werden weitere Definitionen benötigt.

\begin{mydef}\label{2.1}
Sei $(X, \|•\|)$ ein reeller, normierter, linearer Raum und $M\subset X$ mit $M\ne ∅$ sowie $f\colon M\to ℝ$.
\begin{itemize}
    \item Die Menge aller stetigen, linearen Funktionen $f\colon X\to ℝ$ wird mit $\lin (X, ℝ)$ bezeichnet.
    \item Eine Folge $\{ x_k \}_{k\in ℕ}, x_k \in M$ heißt \bemph{schwach konvergent}, wenn ein $\overline x$ existiert, so dass
    $$f(x_k) \to f(\overline x) \qquad ∀f\in \lin(X, ℝ).$$
    Man schreibt $x_k \wto \overline x,k\to \infty$.
    \item Die Menge $M$ heißt \bemph{schwach folgenabgeschlossen}, wenn aus $x_k \in M, x_k \wto \overline x , k\to \infty$ folgt, dass $\overline x\in M$.
    \item Die Menge $M$ heißt \bemph{schwach folgenkompakt}, wenn $M$ schwach folgenabgeschlossen ist und man aus jeder Folge $\{x_k\}\subset M$ eine in $X$ schwach konvergente Teilfolge auswählen kann.
    \item Die Funktion $f\colon M\to ℝ$ heißt \bemph{schwach unterhalb-stetig}, wenn für jede Folge $\{ x_k \}\subset M$ mit $x_k \wto \overline x\in M$ gilt
    $$\liminf_{k\to \infty} f(x_k) \ge f(\overline x).$$
\end{itemize}
\end{mydef}

\begin{bsp}\label{2.2}$ $\\[-1em]
\begin{itemize}
    \item Sei $X = l^p(ℝ) = \{ \{x_k \}\subset ℝ: \sum_{i=0}^\infty |x_k|^p < \infty \}$ mit $1<p<\infty$ und $\{e_j\}_{j\in ℕ}$ mit $e_{j,i} = δ_{j,i}$. Ist $φ\in \lin(X, ℝ)$, so existiert eine Folge $a = \{a_i\} \in l^q( ℝ)$ mit $\frac{1}{p} + \frac{1}{q} = 1$, so dass
$$φ(x) = \sum_{i=0}^\infty a_i x_i \qquad ∀x\in l^p(ℝ).$$
Also $φ(e_j) = a_j \to 0$ und daraus folgt $e_j \wto 0$, während aber $\| e_j \| = 1$ bleibt, also $e_j\not \to 0$.

\item Die Funktion $f\colon ℝ\to ℝ$ mit 
$$f(x) =\begin{cases}
-1 &\text{für $x\in ℕ$},\\
1&\text{sonst.}
\end{cases}$$
ist schwach unterhalb-stetig, aber nicht stetig.
\end{itemize}
\end{bsp}

\begin{folge*}
$ $\\[-1em]
\begin{itemize}
    \item Im $ℝ^n$ entspricht die schwache Konvergenz der starken Konvergenz, d.h.
    $$x_k \wto \ov x ⇔ x_k \to \ov x.$$
    \item Es gibt schwach konvergente Folgen mit dem schwachen Grenzwert $0$, deren Elemente alle die Norm $1$ haben.\\
    Dies zeigt, wie schwach dieser Konvergenzbegriff ist. Deswegen spielt er bei der Entwicklung von Algorithmen keine Rolle, aber bei deren Analyse.
    Ein typisches Resultat ist z.B. „… dann existiert eine schwach konvergente Teilfolge, deren schwacher Grenzwert stationär ist.“
    Außerdem nutzt man ihn bei Existenzaussagen.
\end{itemize}
\end{folge*}
\begin{satz}[Verallgemeinerung von Weierstaß]\label{2.3}
Sei $(X, \| • \|)$ ein reeller, normierter, linearer Raum, $M\subset M, M\ne ∅$ und $f\colon M\to ℝ$. Ist die Menge $M$ schwach folgenkompakt und $f$ schwach unterhalb stetig, so existiert mindestens ein $\ov x \in M$ mit
$$f(\ov x) \le f(x) \qquad ∀x\in M.$$
D.h. das Optimierungsproblem besitzt mindestens eine Lösung.
\end{satz}

\begin{proof}
Sei $\{x_k\}\subset M$ so, dass
$$\lim_{k\to \infty} f(x_k) = \inf_{x\in M} f(x).$$
Da $M$ schwach folgenkompakt ist, existiert eine Teilfolge $\{x_{k_l}\}$ mit $x_{k_l}\wto \ov x\in M, l\to \infty$. Da $f$ schwach unterhalb stetig ist, gilt
$$f(\ov x) \le \liminf_{l\to \infty} f(x_{k_l}) = \inf_{x\in M} f(x),$$
also $f(\ov x) = \min_{x\in M} f(x)$.
\end{proof}

\begin{folge}
Damit erhält man auch die Beschränktheit von $f$ nach unten.
\end{folge}
Der Nachweis der oben definierten Eigenschaften ist allerdings oftmals schwierig. Also betrachten wir noch andere Konzepte, die einfacher nachzuweisen sind!

\begin{mydef}\label{2.4}
Sei $(X, \|•\|)$ wieder ein reeller, normierter Raum, $M\subset X, M\ne ∅$ und $f\colon M\to ℝ$. Die Menge 
$$E(f) \le \{(x,α) \in M \times ℝ : f(x)\le α \}$$
heißt Epigraph von $f$.
\end{mydef}

% skizze 1D epigraph

\begin{satz}\label{2.5}
Sei $(X, \|•\|)$ wieder ein reeller, normierter Raum, $M\subset X, M\ne ∅$ und $f\colon M\to ℝ$. Sei $M$ schwach folgenabgeschlossen. Dann sind folgende Aussagen äquivalent:
\begin{enumerate}[label=\roman*)]
    \item \label{2.5.1} $f$ ist schwach unterhalb stetig.
    \item \label{2.5.2} $E(f)$ ist schwach folgenabgeschlossen.
    \item \label{2.5.3} Ist für ein $α\in ℝ$ die Menge
    $$M_α = \{ x\in M: f(x)\le α \}$$
    nicht leer, so ist $M_α$ schwach folgenabgeschlossen.
\end{enumerate}
\end{satz}

\begin{proof}
Es gelte \ref{2.5.1}. Wir folgern \ref{2.5.2}. Sei also $f$ schwach unterhalb stetig. Sei $\{ (x_k, α_k) \}_{k\in ℕ}$ eine schwach konvergente Folge in $E(f)$ mit dem Grenzwert $(\ov x, \ov α) \in X \times ℝ$. Also 
\begin{align*}
    &x_k \wto \ov x &\text{in $X$}\\
    &α_k \wto \ov α ⇔ a_k \to \ov α &\text{in $ℝ$}.
\end{align*}
Da $M$ schwach folgenabgeschlossen ist, ist $\ov x \in M$.
Sei nun $ε>0$ beliebig. Dann existiert ein $n_0\in ℕ$ mit
$$f(x_k) \le α_k < \ov α + ε \qquad ∀n\ge n_0.$$
Da $f$ schwach unterhalb-stetig ist, folgt
$$f(\ov x) \le \liminf_{k\to \infty} f(x_k) < \ov α + ε.$$
Für $ε \searrow 0$ erhält man
$$f(\ov x) \le \ov α \Rightarrow (\ov x, \ov α) \in E(f).$$

Nun gelte \ref{2.5.2}. Dann folgt \ref{2.5.3}. Sei dazu $α\in ℝ$, so dass $M_α \ne ∅$. Da $M$ schwach folgenabgeschlossen ist, ist auch $M\times \{ α \}$ \sfa und so gilt dies auch für 
$$M_α \times \{ α \} = E(f) \cap \{ M \times \{ α \} \}.$$
Damit ist $M_α$ \sfa.

Es gelte schließlich \ref{2.5.3}. Für einen Widerspruch nehmen wir an, dass für $α\in ℝ$ die Niveaumenge $M_α$ \sfa ist, wenn $M_α\ne ∅$ ist, aber $f$ nicht \suhs ist.
Dann gibt es eine Folge $\{ x_k \}_{k\in ℕ} \subset M$ mit $x_k \wto \ov x$, aber
$$\liminf_{k\to \infty} f(x_k) < f(\ov x).$$
Sei $α\in ℝ$ so, dass
$$\liminf_{k\to \infty} f(x_k) < α < f(\ov x).$$
Dann existiert eine Teilfolge $\{x_{k_l}\}\subset M_α$, die schwach gegen ein $\ov x \in M_α$ konvergiert. Also
$f(\ov x ) \le α$.
\end{proof}

\begin{bem*}
Die Menge $M_α$ heißt auch Levelset oder Niveaumenge.
\end{bem*}
\end{document}