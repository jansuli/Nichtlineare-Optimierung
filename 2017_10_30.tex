\documentclass[main.tex]{subfiles}

\begin{document}
\begin{bem*}$ $\\[-1em]
\begin{itemize}
    \item Die $\Lp$-Räume sind für $1\le p \le ∞$ vollständig mit der angegebenen Norm.
    \item Für $M<p<∞$ sind die $\Lp$-Räume auch reflexiv, denn:
    Jede Funktion $F\in \left( \Lp(Ω) \right)^*$ kann durch eine Funktion $f\in \Larg{q}(Ω)$ mittels der Form
    $$F(x) = \int_Ω f(t) x(t) ~dt$$
    dargestellt werden, wobei $1= \frac1p + \frac1q$.
    Die Stetigkeit von $F$ folgt aus der Hölderschen Ungleichung
    $$\int_Ω |f(t)| |x(t)| ~dt \le \left( \int_Ω |f(t)|^q ~dt \right)^{\frac1q} \cdot \left( \int_Ω |x(t)|^p ~dt \right)^{\frac{1}{p}}$$
    und dem Rieszschen Darstellungssatz.
    Also $\left( \Lp (Ω) \right)^* = L^q(Ω)$ und $\left( \Lp(Ω)\right)^{**} = \Lp(Ω)$.
    \item Für $p=1$ gilt $\left( L^1(Ω)\right) = L^∞(Ω)$, aber
    $$\left( L^∞(Ω) \right)^* \supsetneq L_1 (Ω).$$
\end{itemize}
\end{bem*}

\begin{satz}\label{2.19}
Sei $X$ ein reeller, reflexiver Banachraum und $M\subset X, M\ne ∅$ und $M$ konvex, abgeschlossen sowie beschränkt. Sei $f\colon M\to ℝ$ eine stetige, quasikonvexe Funktion. 
Dann existiert mindestens ein $x\in M$ so, ass $f$ in $x$ minimal wird.
\end{satz}

\begin{proof}
Es gilt: Jede nichtleere, konvexe, abgeschlossene und beschränkte Teilmenge eines reflexiven Banachraums ist schwach folgenkompakt (vgl. \cite{werner}). Dann folgt die Behauptung aus Satz \ref{2.3} mit Lemma \ref{2.12}.
\end{proof}

Eine wichtige Voraussetzung in diesem Satz ist die Stetigkeit von $f$. 
Wir untersuchen deshalb, wann eine Funktion stetig ist. 
Man kann zeigen:
\begin{lemma}\label{2.20}
Sei $(X, \|•\|)$ ein reeller, normierter Vektorraum. Sei $M\subset X, M\ne ∅, M$ offen und konvex.
Ist $f\colon M\to ℝ$ konvex und stetig in einem $\overline{x}\in M$, dann ist $f$ stetig auf ganz $M$.
\end{lemma}

\begin{proof}
Sei $\tilde x\in M$ beliebig. 
Es gilt zu zeigen, dass $f$ stetig ist in $\tilde x$.

Da $M$ offen und $f$ in $\ov x$ stetig ist, gibt es eine abgeschlossene Umgebung
$$U_δ(\ov x) := \{ x\in M: \| x- \ov x\| \le δ \},$$
so dass für $α\in ℝ$ gilt, dass 
$$f(x) \le  α \qquad ∀x\in U_δ(\ov x).$$
Da $M$ konvex und offen ist, existiert ein $λ>1$ mit 
$$\ov x + λ(\tilde x - \ov x) \in M\text{ und } U_{(1-1/λ)δ}(\tilde x) \subset M.$$
%skizze
Für $x\in U_{(1-1/λ)δ}(\tilde x)$ gilt wegen der Konvexität von $f$, dass für ein $y\in U_δ(0_X)$ gilt:
\begin{align*}
    f(x) &= f(\tilde x + (1-λ^{-1}) y )\\
    &= f\left(\tilde x - (1-λ^{-1})\ov x + (1-λ^{-1}) (\ov x + y) \right)\\
    &= f \left( λ^{-1} \underbrace{\left( \ov x +λ(\tilde x - \ov x)  \right)}_{\in M} + (1-λ^{-1})\underbrace{(\ov x + y )}_{\in M} \right) \\
    &\le λ^{-1} f\left( \ov x + λ(\tilde x - \ov x) \right) + (1-λ^{-1}) \underline{f(\ov x + y)}_{\le α}\\
    &\le λ^{-1} f\left( \ov x + λ(\tilde x - \ov x) \right) + (1-λ^{-1})α = β.
\end{align*}
Es gilt also
$$f(x) \le β \qquad ∀x\in U_{(1-λ^{-1})δ}(\tilde x).$$

Sei nun $ε\in (0,1)$ und $x \in U_{(1-λ^{-1})δ}(\tilde x)$. Mit $y\in U_{(1-λ^{-1})δ}(0_X)$ und der Konvexität von $f$ folgt:
\begin{align*}
    f(x) &= f(\tilde x + εy) \\
    &= f\left( (1-ε)\tilde x + ε(\tilde x + y) \right)\\
    &\le (1-ε) f(\tilde x) + ε f( \tilde x + y ) \\
    &\le (1-ε) f(\tilde x) + εβ
\end{align*}
und es folgt
\begin{equation}
\label{eqn:2.20.1}
f(x) - f(\tilde x) \le ε( β - f(\tilde x) ).\tag{$\star$}
\end{equation}

Außerdem gilt
\begin{align*}
f(\tilde x ) &= f \left( \frac{1}{1+ε} (\tilde x + ε y) + \left( 1 - \frac{1}{1+ε}\right) (\tilde x + y) \right) \\
&\le  \frac{1}{1+ε} f(\tilde x + εy) +  \left( 1 - \frac{1}{1+ε}\right) f(\tilde x + y)\\
&\le \frac{1}{1+ε} f(\tilde x + εy) +  \left( 1 - \frac{1}{1+ε}\right)β\\
&\le \frac{1}{1+ε} \left( f(x) + εβ \right),
\end{align*}
also
\begin{align}
    (1+ε) f(\tilde x) & \le f(x) + εβ,\nonumber \\ \Rightarrow\;
    -\left( f(x) - f(\tilde x) \right) & \le ε(β-f(\tilde x))\label{eqn:2.20.2}\tag{$\star\star$}
\end{align}
und somit insgesamt aus \eqref{eqn:2.20.1} und \eqref{eqn:2.20.2}:
$$\left| f(x) - f(\tilde x) \right| \le  ε ( β - f(\tilde x) ) \qquad ∀x\in U_{ε(1-λ^{-1})δ}(\tilde x),$$
also die Stetigkeit von $f$ in $\tilde x$.
\end{proof}
Wir nutzen diese Existenzresultate in den Abschnitten \ref{section:2.3} und \ref{section:2.4}.

\section{Menge von Lösungspunkten}\label{section:2.1}.
\begin{satz}\label{2.21}
Sei $X$ ein reeller Vektorraum und $M\subset X$ nichtleer und konvex. Dann ist für jede quasikonvexe Funktion $f\colon M\to ℝ$ die Menge der Minimalpunkte in $M$ konvex.
\end{satz}
\begin{proof}
Beweis in Übung 4.
\end{proof}

Bislang haben wir globale Minimalpunkte betrachtet. 
Jetzt beschäftigen wir uns mit lokalen Minimalpunkten.

\begin{mydef}\label{2.22}
Sei $(X, \|•\|)$ ein reeller, normierter, linearer Raum und $M\subset X$ nichtleer. 
Ein $\ov x \in M$ heißt lokaler Minimalpunkt von $f$ in $M$, wenn eine Umgebung 
$$U_ε(\ov x) = \{ x\in X: \| x- \ov x \| < ε \}$$
existiert, so dass gilt
$$f(\ov x) \le f(x) \qquad ∀ x\in U_ε(\ov x) \cap M.$$
\end{mydef}
\begin{bem*}
$M$ wurde nicht als offen angenommen!
\end{bem*}
% skizze

\begin{satz}\label{2.23}
Sei $(X, \|•\|)$ ein reeller, linearer, normierter Raum und $M\subset X, M\ne ∅$ eine konvexe Menge. Dann ist jeder lokale Minimalpunkt $\ov x\in M$ einer konvexen Funktion $f\colon M\to ℝ$ auch ein globaler Minimalpunkt von $f$ auf $M$.
\end{satz} 

\begin{proof}
Sei $\ov x\in M$ ein lokaler Minimalpunkt von $f$, dann existiert ein $ε>0$, so dass
$$f(\ov x) \le f(x) \qquad ∀x\in M\cap U_{ε}(\ov x).$$
Sei $x \in M, x\notin U_{ε}(\ov x)$, dann gilt
$$\| x - \ov x \| > ε\;\text{ und }\; λ := \frac12 \frac{ε}{\| x - \ov x \|}\in (0,1).$$
Damit folgt
\begin{align*}
    x_λ &= λx + (1-λ)\ov x \in M\\
    \| x_λ - \ov x \| &= \| λx + (1-λ)\ov x \| = λ \| x- \ov x \| = \frac{1}{2}ε.
\end{align*}
Also $x_λ \in U_{ε}(\ov x) \cap M$ und folglich
\begin{align*}
    f(\ov x) &\le f(x_λ) = f(λx + (1-λ) \ov x ) \\
    &\le λ f(x) + (1-λ) f(\ov x)
\end{align*}
und $f(\ov x) \le f(x)$. 
$\ov x$ ist damit globaler Optimierer.
\end{proof}
Über strikte Konvexität erhält man auch Eindeutigkeit des globalen Minimums.

Das nächste Beispiel ist eine Anwendung der Resultate:
\begin{bsp}\label{2.24}
Wir nehmen Bezug auf Bsp. \ref{1.1}. 
Hier ist $X = ℝ^2$ der reelle und reflexive Banachraum und
$$M = \{ x\in ℝ^2 : 1000 \le x_1^2 x_2 , x_1 \le 3x_2, x_2 \le x_1, 0\le x_1 \le 50, x_2 \ge 0\} \subset X.$$
Dann ist $M$ nicht-leer, konvex und abgeschlossen sowie beschränkt.
Die Zielfunktion 
$f\colon M \to ℝ, \; x= (x_1, x_2) \mapsto lx_1x_2$
ist stetig.
Außerdem gilt
$$f(x_1, x_2 ) = lx_1 x_2 \le α ⇔ x_1x_2 \le \frac{α}{l}$$
und 
$$M_α = \left\{ (x_1, x_2 ) \in ℝ^2 : x_1 x_2 \le \frac{α}{l}\right\}$$
ist nicht konvex, wie man sich graphisch verdeutlicht.
Somit ist Satz \ref{2.19} nicht anwendbar.
Allerdings liefert der Satz on Weierstraß die Existenz von Extrempunkten.
\end{bsp}

\section{Approximationsaufgaben}\label{section:2.3}
Approximationsaufgaben haben die Form:\\
Sei $(X, \|•\|)$ ein reeller normierter Raum und $M\subset X, M\ne ∅$ und $\hat x \in X$ gegeben.
Gesucht ist $\ov x \in M$ mit
$$\| \ov x - \hat x \| \le \| x - \hat x \| \qquad ∀x\in M.$$

\begin{mydef}
\label{2.25}
Sei $(X, \|•\|)$ ein reeller, linearer, normierter Raum. Die Menge $M\subset X,M\ne ∅$ heißt \emph{proximinal}, wenn für jedes $\hat x \in X$ ein $\ov x\in M$ existiert, so dass
$$\| \ov x - \hat x \| \le \| x - \hat x \| \qquad ∀x \in M.$$
Das Element $\ov x$ heißt dann Bestapproximierende von $\hat x$ in $M$.
\end{mydef}
\end{document}