\documentclass[main.tex]{subfiles}

\begin{document}
\begin{abstract}
    In dieser Vorlesung wird nichtlineare Optimierung in unendlich-dimensionalen Räumen (im Funktionenraum) untersucht. Dabei werden auch allgemeine Optimalitätsbedingungen hergeleitet.
\end{abstract}

\chapter{Einleitung}
Wir betrachten hier Probleme der Form:

Sei $(X, \|.\|)$ ein reeller, normierter Vektorraum, $M\subset X$ nichtleer und $f\colon M\to ℝ$. Gesucht ist $x^*\in M$ mit \begin{itemize}
    \item $f(x^*)\le f(x)$ für alle $x\in M$, wenn ein globales Minimum gesucht ist, 
    \item und $f(x^*)\le f(x)$ für alle $x\in M\cap U_ε(x^*)$, falls ein lokales Minimum gesucht ist.
\end{itemize}
Je nach Ausprägung von $X,M$ und $f$ unterscheidet man
\begin{itemize}
    \item unendlich-dimensionale Optimierung, wenn $X$ ein Funktionenraum ist, z.B. $X = C([0,T])$.
    \item endlich-dimensionale Optimierung, wenn $X = ℝ^n$ ist.
    \item kontinuierliche Optimierung, wenn $\inner (M) = \mathring{M} \ne \emptyset.$
    \item diskrete Optimierung, wenn $M\subset ℤ^n$.
\end{itemize}
Wir fokussieren uns auf unendliche-dimensionale und kontinuierliche Optimierung.

\begin{bsp}\label{1.1}[Konstruktion eines Balkens]
Betrachte einen rechteckigen Balken der Länge $l$, Höhe $x_1$ und Breite $x_2$. Das Volumen des Balken soll minimiert werden. Allerdings darf die Spannung im Balken nicht größer als ein vorgegebener Wert sein. Dafür muss gelten
$$1000 \le x_1^2 x_2.$$
Außerdem soll durch
$$x_1 \le 3x_2, \quad x_2 \le x_1$$
eine bestimmte Geometrie des Balken gesichert werden. Zusätzlich soll
$$50 \le x_1 \ge 0, \quad x_2 \ge 0.$$
D.h. wir haben
\begin{align*}
    X &= ℝ^2\\
    M &:= \{ x\in ℝ^2: 1000\le x_1^2 x_2 , x_1 \le 3 x_2, x_2 \le x_1, 50 \ge x_1 \ge 0, x_2 \ge 0\}\\
    f&\colon ℝ^2 \to ℝ, \quad x \mapsto lx_1x_2
\end{align*}
und somit eine endlich-dimensionale, kontinuierliche Optimierungsaufgabe.
\end{bsp}

\begin{bsp}\label{1.2}[Brachistochrone Problem]
Dieses Problem wurde 1696 von Johann Bernoulli untersucht. Gegeben sind dabei zwei Punkte $x_a$ und $x_e$. Welches ist die Kurve, welche die Punkte $x_a$ und $x_e$ verbindet, so dass eine Kugel, die nur durch Schwerkraft bewegt wird, den Punkt $x_e$ in minimaler Zeit erreicht?
%\begin{center}
%\includegraphics[0.4\textwidth]{imgs\Brachistochrone}
%\end{center}

Das Problem lässt sich wie folgt mathematisch formulieren:
Es seien $t_a,t_e \in ℝ, t_e > t_a$ und $\mathcal L\colon (t_a, t_e) \times G \to ℝ$ mit $G \subset ℝ^n \times ℝ^n$, wobei $\mathcal L$ in allen Komponenten zweimal stetig differenzierbar ist. Es sei
\begin{align*}
    X&= \{ x: [t_a, t_e]\to ℝ^n, \text{$x$ zweimal stetig differenzierbar} \}\\
    M &= \{x\in X: x(t_a) = x_a, x(t_e) = x_e \}.
\end{align*}
Gesucht ist dann ein $x\in M$ mit
$$l(x) = \min_{x\in M} \int_{t_a}^{t_e} \mathcal L(t, x(t), \dot x(t) )~dt.$$
Es liegt also ein unendlich-dimensionales Optimierungsproblem vor.
\end{bsp}

\begin{bsp}\label{1.3}[Optimale Steuerung]
Sei $I = [0,1], X= C([0,1])$ und $x\in X$ ein Zustand.
Betrachtet wird
$$\min_{x,u} f(x,u)$$
mit 
$$f(x,u) = \frac{1}{2}\int_0^1 \left( x(t) \right)^2 + \left( u(t) \right)^2~dt = \frac{1}{2} \left( \|x\|_{L^2}^2 + \| u \|_{L^2}^2 \right),$$
so dass die Zustandsgleichung
\begin{align*}
    \dot x(t) &= \frac{1}{2} x(t) + u(t) \\
    x(0) &= 1
\end{align*}
erfüllt ist. Man kann zeigen:\\
Zu jedem vernünftigen $u$ existiert genau ein $x$, welches die Zustandsgleichung erfüllt, d.h. es gilt
\begin{align*}
    x &= x(x) \\
    S(x) &= x & \text{Kontroll-Zustands-Operator}.
\end{align*}
$\min_{x,u} f(x,u)$ ist äquivalent zu $\min_{u} \tilde f(x)$ mit $\tilde f(x) = f(x(u), u)$. Diese Variante nennt man reduzierte Formulierung. Das ist immer noch ein unendlich-dimensionales Optimierungsproblem.
\end{bsp}

\begin{bsp}\label{1.4}
Die hyperbolische Sinus-Funktion $\sinh$ soll auf dem Intervall $[0,2]$ durch eine lineare Funktion $a+bt$ approximiert werden (vgl. Abb. \ref{fig:1.4}). Wie misst man die Approximationsgüte?\\
\begin{figure}[h!]\label{fig:1.4}
    \centering
    \subfile{imgs/bsp14.tex}
    \caption{Lineare Approximation von $\sinh$.}
\end{figure}
Man kann z.B. für $a=0$ die Maximumsnorm wählen
\begin{enumerate}[label=\alph*)]
    \item $$\min_{ b\in ℝ}\max_{t\in [0,2]} | bt - \sinh(t) |.$$
    Das ist ein min-max-Problem und man benötigt Bileveloptimierung.
    Äquivalent dazu ist:
    \item \begin{align*}
        &\min_{b, λ\in ℝ} λ&\text{s.t.}\\
        &λ = \max_{t\in [0,2]} |bt - \sinh(t) |
        \end{align*}
    Dann hat man ein Minimierungsproblem mit Nebenbedingung, welche eine spezielle Struktur hat. Ebenfalls äquivalent ist:
    \item 
    \begin{align*}
        &\min_{b,λ\in ℝ}λ&\text{s.t.}\\
        &\left. \begin{aligned}
        βt - \sinh (t) &\le λ \\
        βt - \sinh(t) &\ge -λ
        \end{aligned}\; \right\} &\text{für alle $t \in [0,2]$}. 
    \end{align*}
    Dieses Problem hat unendlich viele Nebenbedingungen!
\end{enumerate}
\end{bsp}

\begin{bem*}
Im $ℝ^n$ haben wir als Standardnorm die Euklidische Norm
$$\|x\| = \left( \sum x_i^2 \right)^{\frac{1}{2}}$$
und alle Normen sind äquivalent. Das ist im Unendlichdimensionalen \emph{nicht} der Fall! Daher spielt die Wahl der Standardnorm hier eine wichtige Rolle und man muss den Einfluss untersuchen, welchen die Wahl der Norm auf die Optimierung hat.
\end{bem*}

\begin{bsp}\label{1.5}
Seien $a,b \in ℝ$.
\begin{itemize}
    \item Die Menge $C([a,b])$ mit Maximumsnorm
$$\| x\|_{\infty} = \max_{t\in [a,b]} |x(t)|$$
ist ein linearer, normierter Raum. Dieser Raum ist abgeschlossen.
    \item Die Menge $C([a,b])$ mit der $L^2$-Norm
    $$\| x \| = \left( \int_a^b \left( x(t) \right)^2 ~dt \right)^{\frac{1}{2}}$$
    ist ein normierter Raum, aber \emph{nicht} abgeschlossen! \\
    Denn: Seien $a=0,b=2$ und $x_n(t) = \min\{1, t^n\}$. Dann gilt
    \begin{align*}
        \| x_n(t) - x_m (t) \|^2_{L^2} &= \int_0^2 \left( x_n(t) - x_m(t) \right)^2~dt \\
        &= \int_0^1 \left( t^n - t^m \right)^2~dt \\
        &= \int_0^1 t^{2n} - 2t^{n+1} + t^{2m}~dt \\
        &= \frac{1}{2n+1} - \frac{2}{n+m+1} + \frac{1}{2m+1} \\
        &\le \frac{2}{2m+1}&\text{für $m\le n$.}
    \end{align*}
    Das heißt $(x_n)$ ist eine Cauchy-Folge. Für den Grenzwert gilt
    $$
    x(t) = \lim_{n\to \infty} x_n(t) = \begin{cases}
    0&0\le t < 1,\\
    1 & 1\le t \le 2,\end{cases}
    $$
    also $x\notin C( [0,2])$.
\end{itemize}
Dieses Beispiel verdeutigt, dass viele Aussagen von der gewählten Norm abhängen.
\end{bsp}

\begin{bsp}\label{1.6}[Strömung in einem Becken]

\begin{figure}[h!]
    \centering{\subfile{imgs/bsp16.tex}}
\end{figure}

Wir betrachten den Zusammenhang zwischen unendlich-dimensionaler Formulierung und endliche-dimenisonaler Diskretisierung.
Dazu sei ein Wasser-Becken betrachtet, so dass das Wasser an der Oberfläche und am Beckenboden angetrieben werden kann. als mathematische Formulierung ergibt sich
\begin{align*}
    \min J(y,u) &:= \min \frac{1}{2} \int |y_2( 0.4, x_2 )|^2 ~dx_2 + \frac{γ}{2} \| u \|_{L^2}\\
    y &= (y_1, y_2) ~\text{gibt die Geschwindigkeit in $x_1$- bzw. $x_2$-Richtung, \emph{Zustand}}\\
    u &=\text{Kontrolle} 
\end{align*}
Der erste Term der Zielfunktion minimiert die Geschwindigkeit in horizontale Richtung. Der zweite Term minimiert die Kosten für die Kontrolle.\\
$y$ erfüllt die Navier-Stokes-Gleichungen
\begin{align*}
    &\frac{-1}{R_e} \Delta y + (y\cdot \nabla) y + \nabla p = f &\text{in $Ω=[0,1]^2$}\\
    &\dive y = 0&\text{in $Ω$},\\
    &y = b&\text{auf $Γ_u$},\\
    &y = u&\text{auf $Γ_o$},\\
    &y = 0&\text{auf $Γ\setminus \{ Γ_u \cup Γ_o \}$}
\end{align*}
und löst ein Optimierungsproblem mit partiellen Differentialgleichungen.
Als passende Räume wähle $y\in H_0^1(Ω)$ für den Zustand und $u\in L^2(Ω)$ bzw. $u\in L^2(Γ_0)$ für die Kontrolle

Hinsichtlich der \bemph{Optimierung} untersuchen wir eine unendlich-dimensionale Formulierung im Funktionenraum versus einer endlich-dimensionalen Formulierung, also einer partiellen Differentialgleichung mit diskretisierter Kontrolle $u$.\\
Warum tauchen Probleme auf? 
In der unendlich-dimensionalen Formulierung
$$umin J(y,u) = \min J(y(u), u) = \min \tilde J(u)$$
mit $\tilde J(u)$ ausgewertet auf $Γ_0$, $u\in L^2(Γ_0)$ oder sogare $u\in H_0^1(Γ_0) := U$
hat man ein inneres Produkt $\langle •,• \rangle_U$ und eine dadurch induzierte Norm.

In der endich-dimensionalen Formulierung mit diskretisierter Kontrolle $u_h \in ℝ^{n+1}$ nutzt man das übliche Standard-Skalarprodukt und die dadurch induzierte Euklidische Norm im diskretisierten Problem
$$\min_{u_h\in ℝ^{n+1} }\tilde J_h (u_h).$$
Will man eine zu $U$ passende Diskretisierung $U_h$ wählen, ergibt sich als Problem die Wahl der Matrix $T$ in der Überführung der Skalarprodukte
$$\langle u_h, v_h \rangle_{ℝ^n} = u_h^T v_h \stackrel{?}= u_h^T T v_h = \langle u_h, v_h \rangle_{U_h}.$$
In der Regel gilt $T\ne I$! D.h. in der Diskretisierung ist das passende Skalarprodukt wichtig! Einen ähnlichen Effekt beobachtet man auch in der Aerodynamik. \\
Naive Optimierung umgeht man hier mit \bemph{„Hessian Smoothing“}.
\end{bsp}
\end{document}